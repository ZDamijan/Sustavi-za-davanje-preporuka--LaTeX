% Predlozak za pisanje diplomskog rada na PMF-MO
% Opcenita uputstva za LaTeX se mogu npr. naci na 
% http://web.math.hr/nastava/rp3, http://web.math.hr/nastava/s4-prof/latex.pdf

\documentclass[a4paper,oneside,12pt]{memoir} % jednostrano: promijeniti twoside u oneside
% Paket inputenc omogucava direktno unosenje hrvatskih dijakritickih znakova 
% opcija utf8 za unicode (unix, linux, mac)
% opcija cp1250 za windowse
\usepackage[utf8]{inputenc}  % ukoliko se koristi XeLaTeX onda je \usepackage{xunicode}\usepackage{xltxtra}
% Stil za diplomski, unutra je ukljucena podrska za hrvatski jezik
\usepackage{diplomski}
% bibliografija na hrvatskom
\usepackage[languagenames,fixlanguage,croatian]{babelbib} % zahtijeva datoteku croatian.bdf
\usepackage{indentfirst}

% hiperlinkovi 
\usepackage[pdftex]{hyperref} % ukoliko se koristi XeLaTeX onda je \usepackage[xetex]{hyperref}
% Odabir familije fontova:
% koristenjem XeLaTeX-a mogu se koristiti svi fontovi instalirani na racunalu, npr
% \defaultfontfeatures{Mapping=tex-text}
% \setmainfont[Ligatures={Common}]{Hoefler Text}
% ili
% \newcommand{\nas}[1]{\fontspec{Adobe Garamond Pro}\fontsize{24pt}{24pt}\color{Chocolate}\selectfont #1}
% i onda \nas{Naslov ...}
\usepackage{txfonts} % times new roman 
% Paket graphicx sluzi za manipuliranje grafikom 
\usepackage[pdftex]{graphicx} % ukoliko se koristi XeLaTeX onda je \usepackage[xetex]{graphicx}
% Paket amsmath je vec ukljucen
% Dodatno definirane matematicke okoline:
% teorem (okolina: thm), lema (okolina: lem), korolar (okolina: cor),
% propozicija (okolina: prop), definicija (okolina: defn), napomena (okolina: rem),
% slutnja (okolina: conj), primjer (okolina: exa), dokaz (okolina: proof)
% Definirane su naredbe za ispisivanje skupova N, Z, Q, R i C
% Definirane su naredbe za funkcije koje se u hrvatskoj notaciji oznacavaju drukcije 
% nego u americkoj: tg, ctg, ... (\tgh za tangens hiperbolni)
% Takodjer su definirane naredbe za Ker i Im (da bi se razlikovala od naredbe za imaginarni dio kompleksnog broja, naredba se zove \slika).
\usepackage{algorithm2e} % http://ctan.org/pkg/algorithm2e
\usepackage[numbered, framed]{mcode}
\usepackage{indentfirst}
\usepackage{enumitem}
\usepackage{chngcntr}

\renewcommand\thechapter{\arabic{chapter}.}
\renewcommand\thesection{\arabic{chapter}.\arabic{section}.}
\renewcommand{\thethm}{\arabic{chapter}.\arabic{section}.\arabic{thm}}
\renewcommand{\thefigure}{\arabic{chapter}.\arabic{figure}.}
\usepackage[labelsep=space]{caption}
\captionsetup[figure]{labelfont=bf}
\renewcommand{\chaptermark}[1]{\markright{\uppercase{\thechapter~~#1}}}
\renewcommand{\lstlistingname}{Programski kod}

% Podaci o diplomskom radu
\title{Sustavi za davanje preporuka}
\author{Zlatko Damijanić}
\advisor{izv. prof. dr. sc. Luka Grubišić}  % obavezno s titulom
\date{rujan, 2017.}  % oblika mjesec, godina
\dedication{Obitelj, prijatelji, škola, izreka} % posveta

\begin{document}
\renewcommand\thelstlisting{\arabic{chapter}.\arabic{lstlisting}.}

\frontmatter % generira stranicu za potpise povjerenstva, posvetu i sadržaj
% uvod
\begin{intro}
Web, kažu, izlazi iz doba pretrage i ulazi u doba preporuka. Koja je razlika? Pretraga je kada tražimo nešto, a preporuka je kada nešto dobro za što nisi znao da postoji ili nisi znao kako pronaći, pronađe tebe. Informacijsko doba uslijedilo je nakon industrijskog doba i obilježeno je naglim razvitkom i raspostranjenosti informacijske tehnologije pomoću kojih informacije postaju sve dostupnije. Internet i digitalna tehnologija pružaju nove prilike za učenje, omogućuju pojedincima direktnu prodaju ideja, usluga ili proizvoda, pristup informacijama te zadovoljavanje emocionalnih i psiholoških potreba. Povećanjem količine informacije koja nas okružuje javlja se potreba za što boljim upravljanjem informacijama. Bitan aspekt tog upravljanja je filtriranje informacija koja se postiže pretragom ili preporukom. Kod pretrage korisnik upisuje ključne riječi te se na temelju unesenih riječi filtriraju informacije i prikazuju korisniku. Najpoznatiji primjer filtriranja je web tražilica Google koja pretražuje web stranice. Dok preporuke otkrivaju sadržaj na Internetu koji bi mogao biti zanimljiv korisniku na temelju nekih njegovih prethodno pregledanih sadržaja te zatim na temelju tih predviđanja dati mu listu od nekoliko preporuka. Filtriranje se može izvesti i njihovom kombinacijom. Među najpoznatijim primjerima sustava za  preporuku na webu su oni od tvrtke Amazon i Netflix.
% odlomak
\par 
U ovom radu opisan je općeniti sustav za davanje preporuka na webu i njegove zadaće. Dana je njihova podjela i opisani su algoritmi najkorištenijih predstavnika pojedine kategorije. Navedeni su njihovi nedostaci i prednosti te kako ih se može evaluirati.
\end{intro}
% poglavlje
\chapter{Opis i podjela sustava za davanje preporuka na webu}	
\label{Opis i podjela sustava za davanje preporuka na webu}
Sustav za davanje preporuka (eng. recommendation systems) spada u područje filtriranje informacija i čini ga skup metoda s kojima analizira dostupne podatke kako bi predvidio što bi se moglo svidjeti, što bi moglo zanimati ili što je relevantno za korisnika. Najčešći oblik preporuka je lista sadržaja za koje je sustav predvidio da su najrelevantniji za korisnika. Primarni razlog potrebe za takvim sustavima je previše raspoloživih informacija (eng. information overload) i raznih mogćnosti interakcije (eng. interaction overload), na primjer: pregledavanje, kupovanje, označavanje, ocjenjivanje, komentiranje, praćenje itd. Web mjesta kako bi korisnicima olakšali pronalaženje i dostupnost informacija te time povećali posjećnost i profitabilnost implementiraju sustave za davanje preporuka. Područje primjene sustava je veliko što se može vidjeti i iz sljedećih primjera.
% lista
\bigskip
\\ Primjeri web sjedišta koje daju preporuke svojim korisnicima i područje njihove primjene:
\begin{itemize}[topsep=2pt]
\setlength{\parskip}{0pt}
\item movielens.org, netflix.com, imdb.com, themoviedb.org - filmovi
\item last.fm, pandora.com, spotify.com, napster.com - glazba
\item flickr.com, instagram.com, pinterest.com, imgur.com - slike
\item amazon.com, ebay.com, aliexpress.com, alibaba.com - e-trgovina
\item youtube.com, dailymotion.com, vimeo.com, metacafe.com - video
\item del.icio.us, diigo.com, raindrop.io, getpocket.com - web oznake 
\item facebook.com, myspace.com, linkedin.com, gotinder.com - prijatelji
\item airbnb.com, booking.com, trivago.com, tripping.com - smještaj
\item bibsonomy.org, citeulike.org, mendeley.com, zotero.org - publikacije
\item google.com/adsense, propellerads.com, media.net, infolinks.com - oglasi
\end{itemize}
\bigskip
% odlomak
\par
Problem preporuka definiran je 90-tih godina, ali se danas još razvija i poširuje se područje njegove primjene. 2006. godine problem je populariziran natjecanjem "Netflix Prize". Netflix je američko poduzeće koje ima web sjedište za gledanje filmova, serija, emisija i TV kanala na zahtjev. Natjecanje je bilo otvorenog tipa i cilj je bio pronaći algoritam koji će poboljšati predikcije Netflix-ovog algoritma. Trebalo je predvidjeti koju će ocjenu korisnik dodijeliti nekom filmu samo na temelju prijašnjih ocjena korisnika, bez dodatnih informacija o korisnicima i filmovima. Nagradu od 1.000.000 američkih dolara 2009. godine odnio je tim "BellKor's Pragmatic Chaos" koji je poboljšao predikcije za 10.06\%. Netflix je 2010. odine objavio da neće biti drugog natjecanje zbog federalne tužnbe zbog narušavanja privatnosti. Naime, bez obzira što su podaci o korisnicima i filmovima bili anonimni, moglo se povezati pojedine korisnike sa ocijenama filmova na web sjedištu "IMDB". 
% odlomak
\par
Sustave za davanje preporuka dijelimo na nepersonalizirane i personalizirane. Nepersonalizirani sustavi su najednostavniji i ne uzimaju u obzir preferencije korisnike te daju svim korisnicima jednake preporuke. Prednost je što su jednostavni za implementaciju i lako je prikupiti podatke. Zbog širokog spektra područja primjene izraz sadržaj odnosi se na knjige, filmovi, video, proizvodi i ostalo što sustav preporučuje.   
% slika
\begin{figure}[!h]
\begin{center}
\includegraphics[scale=1]{slike/amazon-non_personalized_recom.png}
\caption{Primjeri nepersonaliziranih preporuka na web stranici amazon.com}
\label{fig: amazon nepresonalizirane preporuke}
\end{center}
\end{figure}
\\ Primjeri lista preporuka nepersonaliziranih sustava:
\begin{itemize}[topsep=2pt]
\setlength{\parskip}{0pt}
\item trenutno pregledavane stavke (na vrhu slike \ref{fig: amazon nepresonalizirane preporuke})
\item korisnici koji su kupili neke sadržaje kupili su i sljedeće sadržaje (na sredini slike \ref{fig: amazon nepresonalizirane preporuke})
\item najbolje ocijenjeni sadržaji (proječna ocjena svih korisnika)
\item najnoviji sadržaji
\item najprodavaniji sadržaji
\item najpopularniji sadržaji
\end{itemize}
\bigskip
% odlomak
\par 
Prednost personaliziranih sustava je što analiziraju podatke o korisnicima, njihovim aktivnostima u prošlosti, njihove ocjene proizvoda, njihove odnose s drugim korisnicima, podatke o samom sadržaju i na temelju njih daju personalizirane preporuke za pojedinog korisnika. Kvaliteta sustava mjeri se kroz njegovu sposobnost preporuke sadržaja iz tzv. Long Tail-a. Nepersonalizirani sustavi imaju tendenciju da preporučuju najpopularnije sadržaje, oni postaju sve popularniji i njihov broj se smanjuje. Najpopularniji sadržaji većinom nisu relevanti za pojedinog korisnika. Stoga je važno da sustav daje preporuke ostalih sadržaja kojih je mnogo više, ali koji su relevantni. Slika \ref{fig: long tail} prikazuje odnos tih sadržaja.
% slika
\begin{figure}[!h]
\begin{center}
\includegraphics[scale=0.6]{slike/long_tail.jpg}
\caption{Preporučivanje sadržaja (proizvoda i sl.) iz Long Tail-a (J. Jovanović, 2014.)}
\label{fig: long tail}
\end{center}
\end{figure}
% odlomak
\par 
Najčešći nedostaci personaliziranih sustava su: problem rapršenosti (eng. sparsity), premalo ocjena korisnika na početku rada (eng. cold start), problem novih korisnika ili sadržaja (eng. new user, new item), korisnik ili sadržaj koji nema sličnosti sa niti jednim drugim korisnikom ili sadržajem (eng. black sheep), problem skalabilnosti (eng. scalability), nefleksibilnost (eng. ineflexibility), problem raznolikosti ili pretreniranja (eng. diversity ili overfitting), izolacija preporuka (eng. filter bubble) i problem manipulacije sustava (shilling attacks) te pitanje privatnosti. Objašnjenje navednih nedostataka dan je u \ref{ch: nedostaci personaliziranih sustava} poglavlju.
% odlomak
\par
Prve metode za davanje personaliziranih preporuka bazirane su na ocjenama koje korisnici dodijeljuju sadržajima. Za pojedinog korisnika na temelju njegovih ocjena pronalazi se k-najsličnijih korisnika (eng. k-nearest neighbors, kratica: kNN), za razliku od nepersonaliziranih sustava koji uzimaju u obzir sve ocjene. Predikcija s kojom ocjenom bi promatrani korisnik ocijenio neki saržaj proizlazi agregacijom ocjena njegovih k-najbližih susjeda. Zatim se korisniku ispisuje lista preporuka koja sadrži n sadržaja sa najboljim predikcijama ocjena (eng. top n recommendation). Bez obzira na jednostavnost implementacije i lakoće razumjevanja metode, nije ju moguće koristiti u praksi jer korisnik želi preporuke u stvarnom vremenu. Kako je broj korisnika i sadržaja u praksi vrlo velik onda složenost ovakvih metoda onemogujuće rad u stvarnom vremenu. 
% odlomak
\par
Kako bi se smanjila složenost umjesto da se računa sličnost između korisnika, računa se sličnost između sadržaja te se preporučuju sadržaji koji su najsličniji onima koje je korisnik najbolje ocjenio. Skup sadržaja je u praksi puno manji od skupa korisnika te se u njemu događaju manje promjena što je veću skalabilnost ovoj metodi. Ove dvije metode pripadaju grupi sustava temeljenih na memoriji (eng. memory-based) jer se direktno primjenjuju na podatke bez predprocesiranja. One mogu imati različite implementacije ovisno kojom skalom korisnik ocijenjuje sadržaje, kako mjerimo sličnost između korisnika i na koji način računamo predikciju ocjena.
% odlomak
\par
Prethodno opisane metode kao ulaz koriste surove podatke. Svaki put je potrebno računati predkcije na svim podacima, što usporava sustav. Sustavi temeljeni na modelu (eng. model-based) nad podacima izrađuju model te kasnije ne temelju njega, a ne svih podataka, daje preporuke čime se ubrzava rad sustava. Modeli se izrađuju pomoću algoritama strojnog učenja (eng. machine learning) i rudarenja podacima (eng. data mining). Sustavi temeljeni na memoriji i modelu pripadaju skupini sustava temeljenih na suradnji (eng. collaborative filtering) jer se baziraju na podacima dobivenim zajedničkim radnjama korisnika, kao što su ocijenjivanje, kupovanje, pregledavanje i sl. Njihov formalni opis i primjeri različitih implementacija odrađeni su u \ref{ch: sustavi temeljeni na suradnji} poglavlju.
% odlomak
\par
Kada se sustavi nalazi puno sadržaja, a korisnici ih jako malo ocjenjuju javlja se problem rapršenosti podataka. Zato se razvijaju sustavi temeljeni na sadržaju (eng. content-based). Pomoću funkcija sličnosti (udaljenosti) izračunava se sličnost između sadržaja na temelju njegovih atributa i ključnih riječi. Korisniku se preporučuju najsličniji sadržaji onima koje je korisnik najbolje ocijenio. Primjeri implementacije je dan u \ref{ch: sustavi temeljeni na sadrzaju} poglavlju.
% odlomak
\par
Kada se sustavi primjenjuju na sadržaje koje se vrlo rijetko kupuju i ocjenjuju, kao što su automobili, smještaj, digitalne kamere, tada ne možemo primijeniti gore navedene metode je raspršenost podataka jako mala. Da bi se davale preporuke u takcvim slučajima potrebno je pohraniti sve relevantne podatke o sadržaju. Za takve sustave kažemo da su temeljeni na znanju (eng. knowledge-based). Korisnik može pretraživati sadržaje tako da unosi ograničanje preko tražilice, koja može imati različite oblike unosa. Preporuke moraju zadovoljavati dana ograničenja. Ti sustavi su temeljeni na ograničenjima (eng. constraint-based). U sustavima temeljeni na slučajevima (eng. case based) preporuke se temelje na sličnosti zadanih zahtjeva za pretragu. Navedeni sustavi detaljnije su opisani u \ref{ch: sustavi temeljeni na znanju} poglavlju.
% odlomak
\par
Širenjem društvenih mreža (eng. social web) i spajanjem različitih uređaje na Internet (eng. Internet of thinks) dolazi do razvoja novih metoda preporuka. Metode temeljene na spolu, država, nacionalnosti i sličnih atributa pripadaju sustavima temeljeni na demografiji (eng. demographic-based). Korisnici na društvenim mrežama imaju prijatelje ili pratitelje, preporuke koje se baziraju na takve odnose čine sustave temeljeni na zajednici (eng. social-based ili community-based). Također u takvim mrežama moguće je izraditi profil korisnika te na neki način odrediti njegovu osobnost pa za takve sustave kažemo da su temeljeni na osobnosti. Na nekim web sjedištima omogućeno je korisnicima da slobodno označavaju sadržaj oznakama ili ključnim riječima (eng. folksonomy) što omogućuje izradu sustava temeljene na oznakama (eng. tag-based ili keyword-based). Sustave koje u obzir uzimaju kontekst u kojem se korisnik nalazi (na primjer: lokacija, vrijeme, raspoloženje, dan u tjednu, svrha pretrage) kažemo da se temeljeni na kontekstu. Pregled novih metoda nalazi se u \ref{ch: pregled novih metode} poglavlju.
% odlomak
\par
Kako bi se dobile šro bolje preporuke i savladali nedostaci metode se najčešće kombiniraju te ih nazivamo hibridnim metodama (eng. hybrid methods). Načine povezivanje i primjere možete pronači u
\ref{ch: hibridne metode} poglavlju.
% odlomak
\par
U sustavima koji se temelje na oznakama koje korisnici dodjeljuju sadržajima javlja se problem da neke riječi mogu imati više značenja, više oblika i da više riječi imaju isto značenje. Na koje načine taj problem riješiti i vlastiti doprinos poboljšanju takvih sustava odrađeno je \ref{ch: sustav temeljen na oznakama - ucenje odnosa izmedju oznaka} poglavlju.
% odlomak
\par
Zbog velikog broja metoda, njezini različitih implementacija, kombinacija i mogućih poboljšanja potrebno je razviti nečine kako sustave evaluirati kako bi se pojedinu domenu primjene izabrao najprikladniji. Problem i načini evalucije opisani su u \ref{ch: evalucija sustava} poglavlju, a trenutni izazovi i budućnost sustava za davanje preporuka u \ref{ch: trenutni izazovi i buducnost} poglavlju.
% lista
\bigskip
\\ Sažeti pregled podjele personaliziranih sustava
\begin{itemize}[topsep=2pt]
\setlength{\parskip}{0pt}
\item klasične / tradicionalne metode (eng. classic / traditionals methods)
  \begin{itemize}[topsep=2pt]
  \setlength{\parskip}{0pt}
  \item temeljeni na suradnji (eng. collaborative filtering)
    \begin{itemize}[topsep=2pt]
    \setlength{\parskip}{0pt}
    \item temeljeni na memoriji (eng. memory-based): temeljeni na sličnosti korisnika (eng. user-based ili user-user collaborative filtering), temeljeni na slčnosti objekata (eng. item-based ili item-item collaborative filtering)
    \item temeljeni na modelu (eng. model-based)
    \end{itemize}
  \item temeljeni na sadržaju (content-based)
  \item temeljeni na znanju (eng. knowledge-based)
    \begin{itemize}[topsep=2pt]
	\setlength{\parskip}{0pt}
    \item temeljeni na ograničenjima (eng. constraint-based)
	\item temeljeni na slučajevima (eng. case-based)
	\end{itemize}
  \end{itemize}
\item nove metode (eng. novel methods) - web 2.0 (Social Web), web 3.0 (Internet of Things)
    \begin{itemize}[topsep=2pt]
	\setlength{\parskip}{0pt}
    \item temeljeni na demografiji (eng. demographic-based)
    \item temeljeni na zajednicama (eng. social-based ili community-based)
    \item temeljeni na osobnosti (eng. personality-based)
    \item temeljeni na oznakama (eng. tag-based ili keyword-based)
    \item temeljeni na kontekstu (eng. context-based)
	\end{itemize}
\item hibridne metode (eng. hybrid methods)
\end{itemize}
\bigskip
% slika
\begin{figure}[!h]
\begin{center}
\includegraphics[scale=0.6]{slike/primjer_ocjene.jpg}
\caption{Primjer ocjena koji su korisnici dodijelili pojedinom sadržaju}
\label{fig: primjer ocjena}
\end{center}
\end{figure}

% poglavlje
\chapter{Nedostaci personaliziranih sustava}
\label{ch: nedostaci personaliziranih sustava}
% odlomak
\par 
Da bi bolje razumjelo i opravdalo uvođenje novih i izmjena starih metoda za davanje personaliziranih preporuka potrebno je proučiti moguće nedostake koje sustav može imati. Te nedostatke opisat ćemo na primjeru sa slike \ref{fig: primjer ocjena}. U tom primjeru nalazi se 8  korisnika koji su ocijenili 7 sadržaja.
% odlomak
\par 
Najveći nedostatak je problem raspršenosti. Neki sadražji nisu ili su premalo ocjenjeni dok drugi  su ocjenjeni od većine korinisnika. Na slici \ref{fig: primjer ocjena} možemo vizualno vidjeti kako ocjene nisu dobro raspoređene u matrici. U ovi kategoriju nedostataka spada i problem premalo ocjena na početku rada sustava kad ocjena gotovo i nema i raspršenost je velika. Također problem nastaje kod novih korisnika ili sadržaja jer nemaju nikakve ocjene te je takvom korisniku onemogućeno davanje personaliziranih preporuka, a novi sadžaj se neće nikome preporučiti. Moguće je i da već postojeći korisnici nisu dodijelili ocjene ili su su samo oni dodijeli ocjene nekim sadržajima. Isto tako vrijedi i za postoje sadržaju za koje je moguće da nisu ocijenjeni. Za takve korisnike i sadžaje kažemo da su "crne ovce". Na slici \ref{fig: primjer ocjena} vidimo novog korisnika "Helena" i novi sadržaj "\#7" te crne ovce "Barbaru" i sadržaj "\#3".
% odlomak
\par 
skalabilnost
% odlomak
\par 
nefleksiilnost
% odlomak
\par 
problem raznolikosti ili pretreniranja, izolacija preporuka (eng. filter bubble)
% odlomak
\par 
problem manipulacije sustava
\par 
privatnosti

% poglavlje
\chapter{Sustavi temeljeni na suradnji}
\label{ch: sustavi temeljeni na suradnji}
% odlomak
\par 
Među najpopularnijim algoritmima koji se koriste su oni temeljeni na suradnji. Dijelimo ih na one temeljene na memoriji i one temeljene na modelu. Sustavi temeljeni na modelu koriste sve dostpune podatke bez predprociranja, dok oni temeljni ne modelu na temelju tih podataka prvo izrađuju model pa kasnije preporuke baziraju na tom modelu. U sustavima temeljenim na suradnji korisnici ocjenjuju sadržaje te se na tim ocjenama izrađuju preporuke. U sljedećim sekcijama definirani su osnovni pojmovi koji su nam potrebni za model takvih sustava, prikazani primjeri i implementacije u MatLab-u.
% sekcija
\section{Sustavi temeljeni na memoriji}
\label{sec: sustavi temeljeni na memoriji}
% podsekcija
\subsection{Sustavi temeljeni na sličnosti korisnika}
% odlomak
\par
Ocjena koju bi korisnik dao nekom objektu, kojeg još nije ocijenio, predviđa se na temelju k najsličnijih korisnika koje zovemo k-najbliži susjedi. Način predviđanja ocjena najčešće je određen sa sljedećom definicijom.
% definicija
\begin{defn}
Neka je $U$ skup korisnika (eng. users), $I$ skup objekata (eng. items), $u\in U$ neki korisnik te $i\in I$ neki objekt. Korisnici ocijenjuju pojedni objekt što je dano funkcijom $r:U\times I \rightarrow \N$. Ako korisnik njie ocijenio neki objekt tada je vrijednost funkcije r jednaka nuli. $N_u \subseteq U$ je skup od k korisnika koji su najsličniji korisniku "$u$". Predviđanje koju ocjenu bi dao korisnik "$u$" objektu "$i$" dana je aritmetičkom sredinom s težinama pri čemu su težine zadane s funkcijom sličnosti korisnika $s:U\times U\rightarrow \R$. 
\[ p:U\times I \rightarrow \R^+, p(u,i) = \overline{r_{u}}+\frac{\sum_{u'\in N_{u,i}}s(u,u')(r(u',i)-\overline{r_{u'}})}{\sum_{u'\in N_{u,i}}|s(u,u')|} \]
Pri čemu je $N_{u,i} \subseteq N_i$ podskup od najbližih korisnika koji su ocijenili objekt i, a $\overline{r_{u}}$ aritmetička sredina svih ocjena koje je dao korisnik u.
\[ \overline{r_{u}}=\frac{\sum_{i\in I}r(u,i)}{card\{i\in I | r(u,i) \neq 0 \} } \]
\end{defn}
% odlomak
\par
Iz prethodne definicje vidimo da je za ovaj model moguće na razne načine zadati funkciju sličnosti između korisnika. Niže su navedeni primjeri funkcija sličnosti i kako se one računaju.
\bigskip
\\Primjeri funkcija sličnosti:
\begin{itemize}[topsep=2pt]
\setlength{\parskip}{0pt}
\item Pearsnov koeficijent korelacije (eng. Pearson correlation)
\item Spearmanov koeficijent korelacije (eng. Spearman rank correlation)
\item kosinusova sličnost (eng. cosine similarity)
\end{itemize}
\begin{defn} (Pearsonov koeficijent korelacije)
\\Neka su $u, v\in U$ neki korisnici i $I_u, I_v$ skup objekata koje je ocijenio korisnik u, odnosno v. Pearsnov koeficijent korelacije zadan je sljedećom formulom.
\[ s:U\times U \rightarrow \R, s(u,v) = \frac{\sum_{i\in I_u\cap I_v}(r(u,i)-\overline{r_u})(r(v,i)-\overline{r_v})}{\sqrt[]{\sum_{i\in I_u\cap I_v}(r(u,i)-\overline{r_u})^2} \sqrt[]{\sum_{i\in I_u\cap I_v}(r(v,i)-\overline{r_v})^2} },  \]
pri čemu je $\overline{r_u}$ aritmetička sredina svih ocjena korisnika u i $\overline{r_v}$ je aritmetička sredina svih ocjena korisnika v.
\end{defn}
%
\begin{rem} (Spearmanov koeficijent korelacije)
\\Računa se kao Pearsonov koeficijent korelacije, samo su ulazne vrijednosti rangovi ocjena, a ne same ocjene. Najvećoj ocjeni dodjeljen je rang 1, sve manjim ocjenama dodijeljeni su sve veći rangovi.
\end{rem}
%
\begin{defn} (kosinusova sličnost)
\\Korisnici su reprezentirani s vektorima $|I|$-dimenzionalnim vektorima pri čemi i-ta pozicja predstavlja ocjenu koju je korisnik dao objektu i. Neka su $r_u$ i $r_v$ vektori koji predstavljaju korinike u i v.
\[ s:U\times U \rightarrow \R, s(u,v) = \frac{r_u\cdot r_v}{||r_u||_2 \: ||r_v||_2} = \frac{\sum_i r(u,i)\: r(v,i) }{\sqrt{\sum_i r(u,i)^2}\:\sqrt{\sum_i r(v,i)^2}}  \]
U prilagođenoj kosinusovoj sličnosti uzima se u obzir i prosječna ocjena korisnika.
\[ s(u,v) = \frac{\sum_i (r(u,i)-\overline{r_u})(r(v,i)-\overline{r_v}) }{\sqrt{\sum_i (r(u,i)-\overline{r_u})^2}\:\sqrt{\sum_i (r(v,i)-\overline{r_v})^2}},  \] 
pri čemu je $\overline{r_u}$ aritmetička sredina svih ocjena korisnika u i $\overline{r_v}$ je aritmetička sredina svih ocjena korisnika v.
\end{defn}
% odlomak
\par
a
% podsekcija
\subsection{Sustavi temeljeni na sličnosti sadržaja}
\par 
ovim sustavima se na temelju ocjena korisnika kreira model u kojem je se određuje sličmost između objekata. Za izračun k-najbliži susjeda u potrebno je $O(|U|)$ koraka, moguće i više, ovisno o načinu računanja funkcije sličnosti. U sustavima sa velikim brojem korisnika potreban je brži način predikcije ocjena. Skalabilnost dobivamo ako prethodno izračunamo susjede svakom objektu, smatra se da su sličnosti između ovjekata stabilniji od sličnosti korisnika, pa kod izračuna predikcija koristimo te unaprijed određene susjede. Objekti su sličniji ako imaju podjednake ocjene korisnika. Slično kao prethodnim sustavima, bira se k-najbližih susjeda nekog objekta i na temelju ocjena korisnika tim susjedima i funkcije sličnosti predviđa se ocjena.
\begin{defn}
Neka je $U$ skup korisnika, $I$ skup objekata, $u\in U$ neki korisnik te $i\in I$ neki objekt. Korisnici ocijenjuju pojedni objekt što je dano funkcijom $r:U\times I \rightarrow \N$. $S \subseteq I$ je skup k-najbližih susjeda objekta $i$. Predviđanje koju ocjenu bi dao korisnik $u$ objektu $i$ dana je aritmetičkom sredinom s težinama pri čemu su težine zadane s funkcijom sličnosti objekata $s:I\times I\rightarrow \R$. 
\[ p:U\times I \rightarrow \R^+, p(u,i) = \frac{\sum_{j\in S}s(i,j)r(u,j)}{\sum_{j\in S}|s(i,j)|} \]
\end{defn}
\begin{rem}
U prethodnoj definicije da se izbjegnu predikcije koje mohu biti negativne trebaju se onda uzeti u obzir samo objekti koji imaju nenegativnu sličnost. U sustavima koje je moguće dati samo binarnu ocjenu (0 ili 1) potrebno je drugačije definirati predikciju, može se definirati samo kao suma sličnosti.
\end{rem}
Također i u ovom modelu potrebno je odrediti funkciju sličnosti. Kosinusova sličnost i Pearsonov koeficijent se definira analogno kao u prethodno definiranim sustavim pri čemu se u praksi pokazuje da je puno bolja kosinusova sličnost.
\bigskip
\\Primjeri funkcija sličnosti:
\begin{itemize}[topsep=2pt]
\setlength{\parskip}{0pt}
\item kosinusova sličnost i prilagođena kosinusova sličnost
\item Pearsnov koeficijent korelacije 
\item Spearmanov koeficijent korelacije 
\item uvjetna vjerorajtnost (eng. conditional probability) - koristi se u sustavi s binarnom ocjenom
\end{itemize}
\begin{figure}
\begin{center}
\includegraphics[scale=0.4]{slike/item-based.png}
\caption{Struktura sustava na temelju sličnsti objekata}
\end{center}
\end{figure}
%

% sekcija
\section{Sustavi temeljeni na modelu}
\label{sec: sustavi temeljeni na modelu}
%
Prethodno dva opisana sustava temnelje svoje predikciju na svim ocjenama korisnika, tj. na cijeloj bazi korisnika i objekata pa takve sustave zovemo temeljeni na memoriji. Sustavi temeljeni na modelu kreiraju neki model pomoću algoritama rudarenja podataka  i strojnog učenja (eng. machine learning) nad podacima. Prednost ovakvog pristupa je što se postižu bolja skalabilnost (brže vrijeme izvršavanja i manje memorije kod većih sustava) i postižu se bolje predikcije na rasprešenim podacima. Nedostatatak je taj što je potrebna zahtjevna izrada modela i treba paziti koji se model primjenuje jer dolazi do određenog gubitka podataka.
\bigskip
\\ Neki od algoritama za izradu modela:
\begin{itemize}
\item SVD redukcija dimenzionalnosti (eng. SVD-based reduction dimensionality)
\item analiza glavnih componenti (eng. principle component analysis)
\item asocijacijska pravila
\item klasifikatori: stabla odluke
\item 	Baysove mreže
\item 	neuronske mreže
\item 	SVM
\item grupiranje podataka
\end{itemize}
%
\subsection{SVD redukcija dimenzionalnosti}
\par
Vektori u sustavima temeljeni na memoriju koji predstavljaju korisnika su dimenzija $|I|$, ondnosmo ono koji predstavjaju objekta su dimenzija $|U|$. Kako sustav ima puno korisnika i objekta vektori su velikih dimenzija. Redukcijom dimenzija dobivamo bolje performanse i riješavamo se šuma. Definiramo singularnu dekompoziciju matrice i kako se primjenjuje u preporukama.
\begin{defn} (Dekompozicija singularnih vrijednosti, eng. singular value decomposition, SVD)
\\Neka je M realna matrica, $M\in \R^{m\times n}$. M se može zapisati kao $M=U\Sigma V^*$, pri čemu je U $m\times m$ ortogonalna matrica, a V* $n\times n$ ortogonalna matrica. Matrica $\Sigma$ je $m\times n$ dijagonalna matrica pri čemu su dijagonalni elementi $\sigma_i$ poredani od najvećeg prema najmanjem i zovemo ih singularnim vrijednostima matrice M. Matrica ranga k ima k singularnih vrijednosti, pa je $\sigma_i=0$ za $i>k$.
\\ Stupci matrice U zovemo lijevi singlurani vektori matrice M, a stupce matrice V desni singularni vektori matrice M.
\\ Lijevi singularni vektori jednaki su otronormiranim svojstvenim vektorima matrice MM*, desni singularni vektori jednaki su otronormiranim svojstvenim vektorima matrice M*M, a nenegativne singularne vrijednosti su korjeni nenagativnih svojstvenih vrijednosti matrica MM* (jednake i za M*M).
\\ Kako je M realna matrica tada $V^*=V^T$ i V je ortogonalna tj. $VV^T=1$.
\end{defn}
Neka su u daljnjem tekstu $m:=|U|$ i $n:=|I|$. Prostore s m i n dimenzijama želimo reducirati na prostor dimenzije k, taj prostor ćemo zvati prostor značajki. SVD dekompozicijom otkrivamo k značajki i preferenciju korisniku prikažemo u prostoru značajki te preporučujemo najbliže vektore objekta prikazanih u tom prostoru.
\begin{defn} Neka je R matrica ocjena dimenzija $m\times n$, $r_{i,j}:=r(i,j)$. Matrica ocjena R se može aproksimirati dekompozicijom singularnih vrijednosti $R \approx \hat{U} \hat{\Sigma} \hat{I}^T$. $\Sigma$ je matrica dimenzija $\hat{k}\times \hat{k}$ koja sadrži $\hat{k}$ najveće singularne vrijednosti matrice R, $\hat{k}<k=rang(R)$. Time dobovima preslikavanje vektora korinika i objekta u prostor značajki dimenzije k. Neke je $\hat{u}_l$ l-ti redak matrice $\hat{U}$, $\sigma_l$ l-ti dijagonalni element matrice $\hat{\Sigma}$ i $\hat{i}_l$ l-ti stupac matrice $\hat{I}^T$. Predikcija je dana s:
\[ p:U\times I \rightarrow \R^+, p(u,i) = \sum_{l=1}^k \hat{u}_l \sigma_l \hat{i}_l \]
\end{defn}
\begin{rem} 
Koristeći Frobeniusovu normu kao mjeru greške tada je singularna dekompozicija matrice R najbolja aproksimacija s matricom ranga $\hat{k}$.
\end{rem}
\begin{defn} $\tilde{R}$ je normalizirana matrica matrice R pri čemu je $\tilde{r}_{u,i} = (r_{u,i}-\mu_i)/\sigma_i$. $\mu_i$ je aritmetička sredina ocjena objekta i, a $\sigma_i$ je njihova varijanca.
\end{defn}
\begin{figure}
\begin{center}
\includegraphics[scale=0.8]{slike/svd_decom.png}
\caption{Singularna dekompozicija matrice R}
\end{center}
\end{figure}
%
\subsection{Analiza glavnih komponenti}
Opis!
\begin{defn} (Svojstveni vektor i svojstvena vrijednost)
\\Netrivijalni vektor x dimenzije n ($x\in \R^n$ i $x\neq 0$) zovemo svojstveni vektor realne kvadratne matrice A, $A\in \R^{n\times n}$, ako vrijedi $Ax=\lambda x$ za neki skalar $\lambda \neq 0$. Skalara $\lambda$ ako postoji onda je jedinstven i zovemo ga svojstvena vrijednost matrice A pridružena svojstvenom vektoru x. 
\end{defn}
\begin{defn} (Dekompozicija svojstvenih vrijednosti ili spektralna dekompozicija, eng. eigenvalue decomposition ili spectral decomposition, EDV)
\\Neka je A realna kvadratna matrica, $A\in \R^{n\times n}$, koja ima n linearno nezavisnih svojstvenih vektora $q_i,  i = 1,2,3,...,n.$ Tada se matrica A može zapisati kao $A=Q\Lambda Q^{-1}$, pri čemu je Q matrica sa stupcima koji su jednaki svojstvenim vrijednostima, a $\Lambda$ je dijagonalna matrica gdje je i-ti dijagonalni element jednak svojstvenoj vrijednosti $\lambda_i$ matrice A pridružena svojstvenom vektoru $q_i$.
\\ Ako je A simetrična kvadratna matrica tada se A može zapisati kao $A=Q\Lambda Q^{T}$, gdje je Q ortogonalna matrica.
\end{defn}
\begin{defn}
Neka je $C=\frac{1}{|U|-1}\tilde{R}^T\tilde{R}$ matrica kovarijanci. Matrica C je realna simetrična pozitvno semidefinitna matrica, pa postoji njena spektralna dekompozicija. Analiza glavnih komponenti aproksimira matricu $\tilde{R}$ sa spektralnom dekompozicijom matrice s njenih $\hat{k}$ najvećih svojstvenih vrijednosti, $\tilde{R}\approx \hat{Q}\hat{\Sigma}\hat{Q}^T$. Predikcija je dana s:
\[ p:U\times I \rightarrow \R^+, p(u,i) = ? \]
\end{defn}
%
\subsection{Stabla odluke}
%
\subsection{Baysove mreže}
%
\subsection{Neuronske mreže}
%
\subsection{Markovljevi procesi}
%
\subsection{Grupiranje podataka}

% poglavlje
\chapter{Sustavi temeljeni na sadržaju}	
\label{ch: sustavi temeljeni na sadrzaju}

% poglavlje
\chapter{Sustavi temeljeni na znanju}
\label{ch: sustavi temeljeni na znanju}
% sekcija
\section{Sustavi temeljeni na ograničenjima}
\label{sec: sustavi temeljeni na ogranicenjima}
% sekcija
\section{Sustavi temeljeni na slučajevima}
\label{sec: sustavi temeljeni na slucajevima}

%  poglavlje
\chapter{Pregled novih metoda}	
\label{ch: pregled novih metode}

%  poglavlje
\chapter{Hibridne metode}	
\label{ch: hibridne metode}

%  poglavlje
\chapter{Sustav temeljen na oznakama - učenje odnosa između oznaka}
\label{ch: sustav temeljen na oznakama - ucenje odnosa izmedju oznaka}

%  poglavlje
\chapter{Evaluacija sustava}	
\label{ch: evalucija sustava}

%  poglavlje
\chapter{Trenutni izazovi i budućnost}
\label{ch: trenutni izazovi i buducnost}


% test
\chapter{Ne ulazi u diplomski - Testiranja}	
\begin{algorithm}[H]
\KwIn{
Integers $a \geq 0$ and $b\geq 0$}
\KwOut{\textsc{GCD} of $a$ and $b$}
\While{$b \neq 0$} {
$r \leftarrow a \bmod b$\;
$a \leftarrow b$\;
$b \leftarrow r$\;
}
\caption{Euclidean algorithm}
\end{algorithm}

\begin{lstlisting}[caption=Test]
function [ P ] = kNN_user_based_prediction( R, k, similarity_measure ) 
    m = size(R,1);
    n = size(R,2);
    mean_userRates = sum(R, 2)./sum(R~=0, 2);
    S = similarity_measure( R );
    [~, N] = sort(S,2,'descend');
    N = N(:,1:k);
    P = zeros(m,n);
    for u = 1:m
        for i = 1:n
            if R(u,i) == 0
                P(u,i) = mean_userRates(u) + ...
                dot( (R(N(u,:),i)~=0)' .* S(u,N(u,:)), R(N(u,:),i) - ...
                mean_userRates(N(u,:)) ) / norm( (R(N(u,:),i)~=0)' .* S(u,N(u,:)), 1 );
            else
                P(u,i) = R(u,i);
            end
        end
    end
end
\end{lstlisting}

\lstinputlisting[breaklines=true, caption=Test from file]{kodovi/kNN_user_based_recom.m}

% Na kraju diplomkog rada stavlja se  bibliografija
% Najprije definiramo nacin prikazivanja bibliografije, u ovom slucaju verzija amsplain stila
\bibliographystyle{babplain} % babamspl ili babplain

% U datoteku diplomski.bib se stavljaju bibliografske reference
% Bibliografske reference u bib formatu se mogu dobiti iz MathSciNet baze, Google Scholara, ArXiva, ...
\begingroup
\raggedright
\bibliography{bibliografija/diplomski}
\nocite{*}
\endgroup

\pagestyle{empty} % ne zelimo brojanje sljedecih stranica

% I na koncu idu sazeci na hrvatskom i engleskom

\begin{sazetak}
Ukratko ...
\end{sazetak}

\begin{summary}
In this ...
\end{summary}

% zivotopis
\begin{cv}
Rođen sam 18. kolovoza 1991. godine u Puli. U Rovinju sam završio Osnovnu školu Jurja Dobrila. Već u osnovnoj školi pokazujem interes za matematiku i informatiku posebice za programiranjem te sam sudjelovao na brojnim županijskim i državnim natjecanjima. Bio sam član Udruge tehničke kulture "Galileo Galilei" u Rovinju, a kasnije za udrugu vodim radionicu "Izrada web stranica" za osnovnoškolce. Uspješnim polaganjem državne mature 2010. godine završio sam prirodoslovno-matematičku gimnaziju u Srednjoj školi Zvane Črnje u Rovinju. Iste godine upisujem sveučilišni preddiplomski studij Matematike na Prirodoslovno-matematičkom fakultetu u Zagrebu, zatim 2015. godine  upisujem sveučilišni diplomski studij Računarstva i matematike. Tijekom studija radio sam kao nastavnik matematike i računarstva u Srednjoj školi Zvane Črnje u Rovinju i u Srednjoj strukovnoj školi u Samoboru te sam bio demonstrator iz kolegija Građa računala. Za fakultet redovito sam sudjelovao u sportskim natjecanjima iz dvoranske odbojke i odbojke na pijesku. 
\end{cv}

\end{document}